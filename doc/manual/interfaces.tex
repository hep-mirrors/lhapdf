There are three compiled interfaces to LHAPDF. In order of encouraged usage
they are:
\begin{itemize}
  \item Lhapdf C++ class
  \item C wrapper
  \item Fortran wrapper
\end{itemize}

There is also a python wrapper, which is a simple SWIG wrapper of the C
wrapper. (This project involves a lot of wrapping).

The two "wrapper" interfaces are just that - functions wrapping the core
class with suitable symbol names such that they can be accessed by Fortran
and C respectively. The class interface offers (in the author's opinion)
better named functions and less layers of wrapping - so it is the best way
to go.

The purpose of the LHAPDF class is to create and manage the PDF sets, which
are descended from the class PDF\_Base (described in the next section). One
advantage that this class offers over LHAPDF5 is that it can accept both
absolute and relative paths.

The C wrapper is designed to be completely backwards compatible with
LHAPDF5's C wrapper, providing all the functionality

The list of functions available in the class can be found in Main/Lhapdf.H.

The list of functions available in the C wrapper can be found in 
Main/CCWrap.H

The list of functions avaible to the fortran interface is a complicated
devil FIXME
