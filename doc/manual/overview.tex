This section will attempt to provide an overview of the work done in
LHAPDF6 to improve the memory usage of the library. Rewriting it in C++
gave the opportunity to use dynamic memory allocation functions such as
malloc, in order to only allocate the required amount of memory at run
time. What's more, LHAPDF6 uses dynamic linking to load an evolution
routine meaning that unnecessary code is not loaded into memory.

This report is split into three distinct chapters - the overview, the PDF
specification, the technical details and the cookbooks. For any developer
looking to maintain, the technical details section should cover the
implementation. If you are wanting to add a new evolution routine then
check the cookbooks chapter.

LHAPDF6 is based around a C++ class, with wrappers that provide both a
functional C interface and a fortran interface. The total structure of
LHAPDF6 is shown in figure \ref{fig:classStructure}. This class structure
is explained in great detail in the Technical Details chapter.

\todo[inline]{Redraw this class diagram with the new class hierarchy and TikZ
  (possibly there is an automatic UML diagramming module?)}

\begin{figure}[b]
  \begin{picture}(400,100)
    \setlength{\fboxsep}{1mm}

    %Lhapdf box
    \put(0, 0){\framebox(40, 100){Lhapdf}}
    \put(46, 55){\makebox{creates}}
    \put(40, 50){\vector(1, 0){43}}
    \put(42, 40){\makebox{manages}}

    %PDF classes box
    \put(80, 0)
    {
      \framebox(255,100)[bl]
      {
        \put(100, 85){\framebox{PDF\_Base}}
        \put(128, 82){\vector(0, -1){22}}
        \put(95, 70){\makebox{parent}}
        \put(85, 50){\framebox{Fortran\_PDF\_Base}}
        \put(128, 46){\vector(-1, -1){26}}
        \put(82, 30){\makebox{parent}}
        \put(128, 46){\vector(1, -1){26}}
        \put(148, 30){\makebox{parent}}
        \put(5, 10){\framebox{Fortran\_Evolution\_PDF\vphantom{p}}}
        \put(125, 10){\framebox{Fortran\_Interpolation\_PDF}}

        %Fortran_PDF_Base wraps arrow
        \put(160, 70){\makebox{wraps}}
        \put(173,53){\vector(1, 1){20}}
        \put(190, 60)
        {
          \framebox
          {
            \begin{minipage}{40pt}
              init\_

	      initpdf\_

	      read\_

	      evolve\_

	      alfa\_
	    \end{minipage}
	  }
        }
      }
    }

    \put(340, 0)
    {
	\framebox(100,100)
        {
	  \begin{minipage}{90pt}
	    \begin{center}{PDF\_Info}\end{center}
	    AlphaS info\\
	    Parameter list\\
	    Composition details
	  \end{minipage}
	}
    }
  \end{picture}
  \caption{A diagram of the class structure within LHAPDF6.
           \label{fig:classStructure}}
\end{figure}

\todo[inline]{The rest of things that I want to put here are things like
  examples of speed increases, memory decreases and proof of equality.}
