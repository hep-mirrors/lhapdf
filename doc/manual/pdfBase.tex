PDF\_Base is the polymorphic interface class that the LHAPDF class uses.
Any real PDF sets objects inherit from this providing the appropriate
functions. The general design prospect is for the inherited children to
provide 4 functions, which the PDF\_Base class then manages the calls to.
In the current implementation this is all set up in the Fortran PDF wrapping
but should provide a nice simple interface for C++ PDFs. 

(Please note that there are currently other virtual functions in PDF\_Base
 principally the ones that provide nuclear and photon PDFs. Hopefully in the
 near future these will be wrapped up into the xfx function which is managed
 by PDF\_Base)

The four principal functions to be provided are:

  \begin{verbatim}
    virtual std::vector<double> Evolve(double x, double Q)=0;
    virtual double 		EvolveAlpha(double Q)=0;
    virtual bool Initialise(std::string filename, int member = 0)=0;
    virtual void MakeActive()=0;
  \end{verbatim}

As this is class inheritance, also to be provided are the constructor and
destructor. 

The Initialise function may seem like a strange addition, why have it when
you have a constructor? This is a purely technical issue, relating to the
Fortran implementation. Essentially the Fortran needs to get at the loaded
header data, which it does by virtue of some global functions, which try to
access the active PDF. However during the constructor it is impossible to
make the PDFset the active PDF - essentially the this pointer does not work
properly in the constructor (and you'd need a reference to the LHAPDF
instance that owned it, which we tried to avoid).

MakeActive is also (in my experience) a remnant of having to interface with
Fortran (this will be described in the Fortran\_PDF\_Base section). Suffice
to say, it will probably be unnecessary for any PDFsets written in C++.

Hopefully this provides an easy and adequate way to write general PDF
loading, however currently on our top-down journey we have not standardised
the PDF format. I would have liked to have done this in the PDF\_Base class
but this was not possible. This was because it was necessary to use fortran
file access throughout the entire class, which (without creating a
polymorphic interface that works with both C++ and fortran files) means that
the header loading depends on Fortran. As the creation of C++ PDFs is
a future topic, there was no need for me to either double up the code, or
attempt to write the C/Fortran file interface (which is a silly thing to do,
because anyone writing the actual C code will want to use a standard FILE*
or an ifstream.)
