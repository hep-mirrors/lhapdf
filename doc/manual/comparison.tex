By using MSTW2008cl90_nf3 (continous flavors no subgrids) and NNPDF2008(something)
we compared the accuracy of our general bicubic interpolator to either 
of the set specific interpolators. (Extrapolators have been ingored 
for the moment)

The goal of our interpolation is to have a shortest path between all 
points with in C-infinity. Meaning we have a requirment of having 
hopefully a simple interpolated function in the end. As such polynomial 
interpolation falls straight away because of runge's phenonema. 
(Extreme frequencies with many points)

Two tests to compare our interpolations were:
1) Simply, xfxv6 - xfxv5, and relative error... Showed us how we 
differ from each other...

However since both are interpolating and both are just giving their 
version of "wanted" function should be we had to come up with a better 
test of accuracy. Hence the definition of a good interpolation.

Linear interpolation is the shortest path between the two points, 
with a discontinuous first derivative (C0 I think). 

let value xfxl be the linear interpolation of (X,Q2)

we compared xfxv6 and xfxv5 by:
xfxl - xfxv6
xfxl - xfxv5

relative errors, we get a measure of how far each interpolation strayed 
from the ideal path, while both interpolations are C2...

We need to have color/heat graphs which Dave and I produced near the 
end, sadly this was rather informal and not well documented, should be 
worked on alittle, before we really say we had a better overall 
interpolation.
