As in previous versions of LHAPDF the basic usage of the library is made
straightforward.

\todo{Much of this is now defunct: Andy to rewrite}

\subsection{Loading a PDF}
By name (looks in standard lhapdf directory)

\red{\code{PDFSet::loadByName} or \code{PDFSet::load} will both return a
  \code{PDFSet} pointer which is entrusted to the user to clean up when it is no
  longer needed. A simple delete will clear all memory used. The returned PDFSet
  will contain all metadata global to the entire set. This can be queried
  without loading any PDF data and therefore is very quick to handle.}

\red{Once a PDF needs to get loaded it is done simply by a call to
  PDFSet.getMember, this will return a reference (hopefully indicating that
  memory is handled by the PDFSet class).}

The PDF interface was written in such a way that it is agnostic to the
underlying structure of the data. Currently only PDFGrid is implemented to be a
single grid representation of the PDF function. However when a different
requirements arise of a PDF, it is fairly trivial to implement another class
which will be covered in a later section.

\red{If working on a memory-restricted system, espacially with large PDF sets such as
NNPDF, it is possible to tell the PDFSet that a certain member is no longer
required and therefore cleared from memory freeing it up for another member to
be loaded.}


\subsection{Using custom \code{Interpolator}/\code{Extrapolator} objects}
This section is specific to \code{GridPDF}.

The \code{PDF} interface defines that the \xf functions are transparent to the
user and simply provide the required information. For the most used use case
this is fine, however is specific situations one might want to use different
interpolators or extrapolators to working on grids instead of the default
interpolators which are instantiated automatically by the members.

% The returned \code{PDF} must be dynamic_cast to a PDFGrid* to be able to use
% specific functions provided only by PDFGrid and not through \code{PDF} since they would
% be out of place.

4) Calculate value at \xqsq When a \code{PDF} is held the value of all available
flavors can be evaluated at any \xqsq within physical limits. Physical limits
are for example $x \in [0,1]$ and $Q \ge 0$: values outside this range are not allowed as they
do not make physical sense.
