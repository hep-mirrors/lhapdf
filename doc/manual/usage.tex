As in previous versions of LHAPDF the basic usage of the library is made 
straightforward.

1) Loading a PDFSet
-Multiple options:
	- By Path
	- By Name (looks in standard lhapdf directory)

PDFSet::loadByName or PDFSet::load will both return a PDFSet pointer which is 
entrusted to the user to clean up when it is no longer needed. A simple delete 
will clear all memory used. The returned PDFSet will contain all meta data 
global to the entire set. This can be queried without loading any PDF data and 
therefore is very quick to handle.

2) Loading a PDF
Once a PDF needs to get loaded it is done simply by a call to PDFSet.getMember, 
this will return a reference (hopefully indicating that memory is handled by 
the PDFSet class).

The PDF interface was written in such a way that it is agnostic to the 
underlying structure of the data. Currently only PDFGrid is implemented to be a 
single grid representation of the PDF function. However when a different 
requirements arise of a PDF, it is fairly trivial to implement another class 
which will be covered in a later section.

If working on a memory restricted system, espacially with large PDF sets such as 
NNPDF, it is possible to tell the PDFSet that a certain member is no longer 
required and therefore cleared from memory freeing it up for another member to 
be loaded.

3) Using custom Interpolators/Extrapolators
This section is specific to PDFGrid.

The PDF interface defines that the xfx functions are transperent to the user and 
simply provide the required information. For the most used use case this is 
fine, however is specific situations one might want to use different 
interpolators or extrapolators to working on grids instead of the default 
interpolators which are instantiated automatically by the members.

The returned PDF& must be dynamic_cast<> to a PDFGrid* to be able to use 
specific functions provided only by PDFGrid and not through PDF since they would 
be out of place.

4) Calculate value at (X,Q2)
When a PDF& is held the value of all available flavors can be evaluated 
at any (X,Q2) within physical limits. Physical limits are for example X 
< 0 is not allowed, X > 1 is not allowed, because these don't make 
sense.
